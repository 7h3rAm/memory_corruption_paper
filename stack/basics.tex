\subsection{Basics}
At the heart of the stack-based buffer overflow is the buffer overflow. This usually 
occurs when data is being copied into an array, but proper bounds-checking is not performed,
allowing writing to sections of memory not belonging to the array. For example, the
following code contains a buffer overflow vulnerability:
\lstinputlisting[language=C]{stack1.c}
This will locally allocate 16 bytes for our character array. Conceptually, these 16 bytes
are often thought of as separate from other areas of memory, but after compiling and disassembling,
we can see that this is plainly not the case:
\begin{lstlisting}
080483fc <main>:
80483fc:       55                      push   %ebp
80483fd:       89 e5                   mov    %esp,%ebp
80483ff:       83 e4 f0                and    $0xfffffff0,%esp
8048402:       83 ec 30                sub    $0x30,%esp
8048405:       c7 44 24 2c f0 f0 f0    movl   $0xf0f0f0f0,0x2c(%esp)
804840c:       f0 
804840d:       8d 44 24 1c             lea    0x1c(%esp),%eax
8048411:       89 04 24                mov    %eax,(%esp)
8048414:       e8 b7 fe ff ff          call   80482d0 <gets@plt>
8048419:       c9                      leave  
804841a:       c3                      ret    
\end{lstlisting}

\small{(note: this example used gcc 4.7.1, objdump 2.22.0, and gdb 7.4.1)}\normalsize\\

First we have the usual function prelude, followed by an AND'ing of the \texttt{\$esp} register with \texttt{0xfffffff0}.
At line 5 we grow the stack by \texttt{0x30} bytes, and from lines 6 and 8, we can see that our ``flag" integer
is at \texttt{\$esp+0x2c}, while our character array begins at \texttt{\$esp+0x1c}. If we were to pause execution
immediately before the call to \texttt{gets()} on line 10, the portion of interest of our stack would look as follows:

%\begin{table}
\begin{center}
	\begin{tabular}{r|l|}
	higher memory addresses & \vdots\\
	\hline\hline
	\multirow{4}{*}{} & \texttt{0xf0}\\\cline{2-2}& \texttt{0xf0}\\\cline{2-2}& \texttt{0xf0}\\\cline{2-2}
	flag & \texttt{0xf0}\\
	\hline
	% this makes the figure a bit too big. \multirow{16}{*}{} & \\\cline{2-2}& \\\cline{2-2}& \\\cline{2-2}& \\\cline{2-2}& \\\cline{2-2}& \\\cline{2-2}& \\\cline{2-2}& \\\cline{2-2}& \\\cline{2-2}& \\\cline{2-2}& \\\cline{2-2}& \\\cline{2-2}& \\\cline{2-2}& \\\cline{2-2}& \\\cline{2-2}
	array (16 uninitialized bytes)&\vdots \\\cline{2-2}
	\hline
	24 unused bytes & \vdots \\\cline{2-2}
	\hline
	\multirow{3}{*}{} &\\\cline{2-2} &\\\cline{2-2} (address of array)&\\\cline{2-2}
	\texttt{\$esp} $\rightarrow$ &\\
	\hline\hline
	lower memory addresses & \vdots\\
	\end{tabular}
%\caption{stack during execution}
\end{center}
%\end{table}

Normally this would just be an implementation detail that the compiler abstracts away for
the user, but in the case of buffer overflow vulnerabilities, such details become important.
On line 7 of our C source, we have a call to the standard library function \texttt{gets}. A quick read
of the manpage reveals the notoreity \texttt{gets} has earned itself for its role in buffer overflows:\\

%\begin{lstlisting}
\begin{sloppypar} %fix to keep from overrunning the margin
\texttt{
Never use gets().  Because it is impossible to tell without knowing the
data in advance how many  characters  gets()  will  read,  and  because
gets() will continue to store characters past the end of the buffer, it
is extremely dangerous to use.  It has  been  used  to  break  computer
security.  Use fgets() instead.
}\\
\end{sloppypar}
%\end{lstlisting}

When even the man page so strongly advises against use, we can appreciate the severity of the flaw.
However, nothing is quite like a hands-on example. In fact, let's debug the program under the influence of this bug.
We'll just do something simple, changing the value stored in the local integer ``flag". This would normally
be impossible to do from just operating on a different variable, but the lack of bounds-checking in \texttt{gets}, as 
well as the integer's position behind our buffer on the stack allow this to happen:

\begin{lstlisting}
[zandi@hacktop paper]$ gdb -q ./stack1
Reading symbols from /home/zandi/OU/exploits/paper/stack1...(no debugging symbols found)...done.
(gdb) disas main
Dump of assembler code for function main:
 0x080483fc <+0>:     push   %ebp
 0x080483fd <+1>:     mov    %esp,%ebp
 0x080483ff <+3>:     and    $0xfffffff0,%esp
 0x08048402 <+6>:     sub    $0x30,%esp
 0x08048405 <+9>:     movl   $0xf0f0f0f0,0x2c(%esp)
 0x0804840d <+17>:    lea    0x1c(%esp),%eax
 0x08048411 <+21>:    mov    %eax,(%esp)
 0x08048414 <+24>:    call   0x80482d0 <gets@plt>
 0x08048419 <+29>:    leave  
 0x0804841a <+30>:    ret    
End of assembler dump.
(gdb) b *(main+29)
Breakpoint 1 at 0x8048419
(gdb) r
Starting program: /home/zandi/OU/exploits/paper/stack1 
warning: Could not load shared library symbols for linux-gate.so.1.
 you need "set solib-search-path" or "set sysroot"?
asdfasdfasdfasdfAAAA

Breakpoint 1, 0x08048419 in main ()
(gdb) x/x ($esp+0x2c)
0xbffff9bc:     0x41414141
(gdb) x/s ($esp+0x1c)
0xbffff9ac:      "asdfasdfasdfasdfAAAA"
\end{lstlisting}
As we can see, we filled our buffer with more than its limit of 16 characters, carefully choosing the
4 characters to overflow with to have hex code \texttt{0x41} with standard ANSI encoding. This makes confirmation
of overwriting our integer easy, and we can see that indeed we have changed the local integer ``flag" from
\texttt{0xf0f0f0f0} to \texttt{0x41414141}.\\

So, we see how lack of proper bounds-checking on such operations can lead to unintended consequences,
but how bad can it really be? It may be hard to imagine a situation where overwriting a variable like
this can be dangrous, but with a little ingenuity, it's easy to see.\\


